\documentclass[12pt,english,a4paper]{article}
\pdfobjcompresslevel=0
\usepackage[usenames,dvipsnames]{xcolor}
\usepackage[includeheadfoot,margin=0.8 in,top=0.6 in]{geometry}
\usepackage{siunitx,physics,cancel,upgreek,varioref,listings,booktabs,tocloft, pdfpages}
\usepackage{mathtools}
\usepackage{babel}
\usepackage{graphicx}
\usepackage{float}
\usepackage{fouriernc}
\usepackage{fancyhdr}
\usepackage[utf8]{inputenc}
\usepackage{amsmath}
\usepackage{amssymb}
\usepackage{textcomp}
\usepackage{lastpage}
\usepackage{microtype}
\usepackage{ifthen}
\usepackage{longtable}
\usepackage[linktoc=all, bookmarks=true, pdfauthor={Anders Johansson}]{hyperref}
\renewcommand{\CancelColor}{\color{red}}
\renewcommand{\exp}[1]{\mathrm{e}^{#1}}
\newcommand{\R}{\mathbb{R}}
\newcommand{\tittel}[1]{\title{#1 \vspace{-7ex}}\author{}\date{}\maketitle\thispagestyle{fancy}\pagestyle{fancy}\setcounter{page}{1}}

\newcommand{\deloppg}[2][]{\subsection*{#2) #1}\addcontentsline{toc}{subsection}{#2)}\refstepcounter{subsection}\label{#2}}
\newcommand{\oppg}[1]{\section*{Oppgave #1}\addcontentsline{toc}{section}{Oppgave #1}\refstepcounter{section}\label{oppg#1}}

\labelformat{section}{section~#1}
\labelformat{subsection}{section~#1}
\labelformat{subsubsection}{paragraph~#1}
\labelformat{equation}{equation~(#1)}
\labelformat{figure}{figure~#1}
\labelformat{table}{table~#1}

\lstset{rangeprefix=/*\#,
rangesuffix=\#*/,
includerangemarker=false}
\renewcommand{\lstlistingname}{Code snippet}
\definecolor{codegreen}{rgb}{0,0.6,0}
\definecolor{codegray}{rgb}{0.5,0.5,0.5}
\definecolor{codepurple}{rgb}{0.58,0,0.82}
\definecolor{backcolour}{rgb}{0.95,0.95,0.92}
\lstset{showstringspaces=false,
basicstyle=\footnotesize\ttfamily,
keywordstyle=\color{codegreen},
commentstyle=\color{magenta},
numberstyle=\tiny\color{codegray},
stringstyle=\color{codepurple},
frameshape={RYRYNYYYY}{yny}{yny}{RYRYNYYYY},
breaklines=true,
%literate={0}{{\textcolor{blue}{0}}}{1}%
%             {1}{{\textcolor{blue}{1}}}{1}%
%             {2}{{\textcolor{blue}{2}}}{1}%
%             {3}{{\textcolor{blue}{3}}}{1}%
%             {4}{{\textcolor{blue}{4}}}{1}%
%             {5}{{\textcolor{blue}{5}}}{1}%
%             {6}{{\textcolor{blue}{6}}}{1}%
%             {7}{{\textcolor{blue}{7}}}{1}%
%             {8}{{\textcolor{blue}{8}}}{1}%
%             {9}{{\textcolor{blue}{9}}}{1}%
%             {.0}{{\textcolor{blue}{.0}}}{2}% Following is to ensure that only periods
%             {.1}{{\textcolor{blue}{.1}}}{2}% followed by a digit are changed.
%             {.2}{{\textcolor{blue}{.2}}}{2}%
%             {.3}{{\textcolor{blue}{.3}}}{2}%
%             {.4}{{\textcolor{blue}{.4}}}{2}%
%             {.5}{{\textcolor{blue}{.5}}}{2}%
%             {.6}{{\textcolor{blue}{.6}}}{2}%
%             {.7}{{\textcolor{blue}{.7}}}{2}%
%             {.8}{{\textcolor{blue}{.8}}}{2}%
%             {.9}{{\textcolor{blue}{.9}}}{2}%
}

\renewcommand{\footrulewidth}{\headrulewidth}
\tocloftpagestyle{fancy}

\setcounter{secnumdepth}{4}
\renewcommand{\thesection}{\arabic{section}}
\renewcommand{\thesubsection}{\arabic{section}.\arabic{subsection}}
\renewcommand{\thesubsubsection}{\arabic{section}.\arabic{subsection}.\arabic{subsubsection}}
\setlength{\parindent}{0cm}
\setlength{\parskip}{1em}

\newcommand{\eqtag}[1]{\refstepcounter{equation}\tag{\theequation}\label{#1}}
\hypersetup{colorlinks=true,urlcolor=blue,linkcolor=black}

\sisetup{detect-all}
\sisetup{exponent-product = \cdot, output-product = \cdot,per-mode=symbol}
\sisetup{output-decimal-marker={.}}
\sisetup{round-mode = off, round-precision=3}
\sisetup{number-unit-product = \ }
\DeclareSIUnit\year{yr}

\allowdisplaybreaks[4]
\fancyhf{}

\rhead{Anders Johansson}
\rfoot{Page \thepage{} of \pageref{LastPage}}
\lhead{FYS3150}
%
\usepackage[backend=biber,citestyle=numeric-comp,bibstyle=numeric,sorting=none]{biblatex}
\DefineBibliographyStrings{norsk}{%
  bibliography = {Referanser},
}
\DefineBibliographyStrings{english}{%
  bibliography = {References},
}
\addbibresource{kilder.bib}

\newcommand{\program}[1]{\href{https://github.com/anjohan/Offentlig/blob/master/FYS3150/Oblig4/#1}{#1}}

\newcommand{\gray}[1]{\textcolor{gray}{#1}}
\newcommand{\spin}[1]{\ifthenelse{#1 = 1}{\uparrow}{\downarrow}}
\newcommand{\tilstand}[4]{
    \(\displaystyle
        \begin{matrix}
                & \gray{\spin{#3}} & \gray{\spin{#4}}\\
            \gray{\spin{#2}} & \spin{#1}  & \spin{#2} & \gray{\spin{#1}}\\
            \gray{\spin{#4}} & \spin{#3} & \spin{#4} &  \gray{\spin{#3}}\\
                & \gray{\spin{#1}} & \gray{\spin{#2}}
        \end{matrix}
    \)
}

\title{FYS3150 Project 4}
\author{Anders Johansson}
\begin{document}
%\includepdf{forside.pdf}
\maketitle
\pagestyle{fancy}
\tableofcontents

\begin{abstract}
\end{abstract}

\section{Introduction}

\section{Theory}

\subsection{Example: Analytical expression for the \(2\times2\) case}
For a \(2\times2\) lattice, the thermodynamic quantities can be found analytically without too much work. To find the partition function, we need to write out all possible microstates and calculate their energies. Using the periodic boundary conditions, we get
\begin{longtable}{ccc}\toprule
Microstate & Energy & Magnetization\\ \midrule
\endfirsthead
\midrule
Microstate & Energy & Magnetization\\ \midrule
\endhead
\bottomrule
\endlastfoot
    \tilstand{0}{0}{0}{0} & \(E = -8J\) & \(M = -4\)\\ \midrule
    \tilstand{0}{0}{0}{1} & \(E = 0\) & \(M = -2\)\\ \midrule
    \tilstand{0}{0}{1}{0} & \(E = 0\) & \(M = -2\)\\ \midrule
    \tilstand{0}{0}{1}{1} & \(E = 0\) & \(M = 0\)\\ \midrule
    \tilstand{0}{1}{0}{0} & \(E = 0\) & \(M = -2\)\\ \midrule
    \tilstand{0}{1}{0}{1} & \(E = 0\) & \(M = 0\)\\ \midrule
    \tilstand{0}{1}{1}{0} & \(E = 8J\) & \(M = 0\)\\ \midrule
    \tilstand{0}{1}{1}{1} & \(E = 0\) & \(M = 2\)\\ \midrule
    \tilstand{1}{0}{0}{0} & \(E = 0\) & \(M = -2\)\\ \midrule
    \tilstand{1}{0}{0}{1} & \(E = 8J \) & \(M = 0\)\\ \midrule
    \tilstand{1}{0}{1}{0} & \(E = 0 \) & \(M = 0\)\\ \midrule
    \tilstand{1}{0}{1}{1} & \(E = 0 \) & \(M = 2\)\\ \midrule
    \tilstand{1}{1}{0}{0} & \(E = 0 \) & \(M = 0\)\\ \midrule
    \tilstand{1}{1}{0}{1} & \(E = 0 \) & \(M = 2\)\\ \midrule
    \tilstand{1}{1}{1}{0} & \(E = 0 \) & \(M = 2\)\\ \midrule
    \tilstand{1}{1}{1}{1} & \(E = -8J \) & \(M = 4\)\\
\end{longtable}

To summarise, we have
\begin{table}[H]
    \centering
    \begin{tabular}{cccc}\toprule
        Number of \(\uparrow\) & Multiplicity & Energy & Magnetisation\\ \midrule
        \(4\) & \(1\) & \(-8J\) & \(4\) \\ \midrule
        \(3\) & \(4\) & \(0\) & \(2\) \\ \midrule
        \(2\) & \(2\) & \(8J\) & \(0\) \\ \midrule
        \(2\) & \(4\) & \(0\) & \(0\) \\ \midrule
        \(1\) & \(4\) & \(0\) & \(-2\) \\ \midrule
        \(0\) & \(1\) & \(-8J\) & \(-4\) \\ \bottomrule
    \end{tabular}
\end{table}

Summing over all microstates, we get the partition function
\[
    Z = \sum_{\substack{\text{all}\\ \text{micro-}\\ \text{states}}} \exp{-\beta E_i}
    = 2\exp{8J\beta} + 2\exp{-8J\beta} + 12
\]

The expectation value of the energy can then be found from
\begin{alignat*}{2}
    \langle E\rangle &= \pdv{\ln(Z)}{\beta} = \pdv{\beta}(\ln(2\exp{8J\beta} + 2\exp{-8J\beta} + 12))\\
    &= \frac{1}{2\exp{8J\beta} + 2\exp{-8J\beta} + 12}\qty(16J\exp{8J\beta}-16J\exp{-8J\beta})\\
    &= \frac{8J\qty(\exp{8J\beta}-\exp{-8J\beta})}{\exp{8J\beta}+\exp{-8J\beta}+6}
\end{alignat*}
yielding the heat capacity
\begin{alignat*}{2}
    C_V &= \pdv{T}(\langle E\rangle) = \pdv{\beta}{T}\pdv{\beta}(\langle E\rangle)=-\frac{1}{kT^2}\pdv{\beta}(\langle E\rangle)\\
    &= -\frac{1}{kT^2}\pdv{\beta}(\frac{8J\qty(\exp{8J\beta}-\exp{-8J\beta})}{\exp{8J\beta}+\exp{-8J\beta}+6})\\
    &= -\frac{8J}{kT^2}\frac{8J\qty(\exp{8J\beta}+\exp{-8J\beta})(\exp{8J\beta}+\exp{-8J\beta}+6)-\qty(\exp{8J\beta}-\exp{-8J\beta})8J\qty(\exp{8J\beta}-\exp{-8J\beta})}{\qty(\exp{8J\beta}+\exp{-8J\beta}+6)^2}\\
    &= -\frac{64J^2}{kT^2}\frac{6\qty(\exp{8J\beta}+\exp{-8J\beta})+\qty(\exp{8J\beta}+\exp{-8J\beta})^2-\qty(\exp{8J\beta}-\exp{-8J\beta})^2}{\qty(\exp{8J\beta}+\exp{-8J\beta}+6)^2}\\
    &= -\frac{64J^2}{kT^2}\frac{6\qty(\exp{8J\beta}+\exp{-8J\beta})+\exp{16J\beta}+2+\exp{-16J\beta}-\qty(\exp{16J\beta}-2+\exp{-16\beta})}{\qty(\exp{8J\beta}+\exp{-8J\beta}+6)^2}\\
        &= -\frac{64J^2}{kT^2}\frac{6\qty(\exp{8J\beta}+\exp{-8J\beta})+4}{\qty(\exp{8J\beta}+\exp{-8J\beta}+6)^2}
\end{alignat*}
To find the various quantities connected to magnetization, we use the general formula
\[
\langle A\rangle = \frac{1}{Z}\sum_{\substack{\text{all}\\ \text{micro-}\\ \text{states}}}A_i\exp{-\beta E_i}
\]
Mean magnetic moment(s):
\begin{alignat*}{2}
    \langle M\rangle &= \frac{1}{2\exp{8J\beta} + 2\exp{-8J\beta} + 12}\qty(-4\exp{8J\beta}+4\cdot(-2)+0+0+4\cdot2+4\exp{8J\beta})=0\\
    \langle M^2\rangle &= \frac{1}{2\exp{8J\beta} + 2\exp{-8J\beta} + 12} \qty(\qty(-4)^2\exp{8J\beta}+4\cdot\qty(-2)^2+0+0+4\cdot2^2+4^2\exp{8J\beta})\\
    &= \frac{32\qty(\exp{8J\beta}+1)}{2\exp{8J\beta} + 2\exp{-8J\beta} + 12}
    = \frac{16\qty(\exp{8J\beta}+1)}{\exp{8J\beta} + \exp{-8J\beta} + 6}\\
    \langle \abs{M}\rangle &= \frac{1}{2\exp{8J\beta} + 2\exp{-8J\beta} + 12}\qty(\abs{-4}\exp{8J\beta}+4\cdot\abs{-2}+0+0+4\cdot2+4\exp{8J\beta})\\
    &= \frac{8\qty(\exp{8J\beta}+2)}{2\exp{8J\beta} + 2\exp{-8J\beta} + 12}
    = \frac{4\qty(\exp{8J\beta}+2)}{\exp{8J\beta} + \exp{-8J\beta} + 6}
    \intertext{Magnetic susceptibility:}
    \chi &= \beta\qty(\langle M^2\rangle - \cancel{\langle M\rangle^2})=
    \frac{16\beta\qty(\exp{8J\beta}+1)}{\exp{8J\beta} + \exp{-8J\beta} + 6}
\end{alignat*}











\clearpage
\addcontentsline{toc}{section}{References}
\printbibliography


\end{document}
